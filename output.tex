\RequirePackage{silence}
\WarningFilter{biblatex}{Patching footnotes failed}
\documentclass[10pt, compress,british,xcolor={svgnames,dvipsnames,x11names},trans]{beamer}

\usepackage{babel}
\usepackage{csquotes}
\usepackage{comment}
\usepackage{tikzsymbols}

%%% mtheme customisations
\usetheme[progressbar=frametitle,block=fill]{m}
\AtBeginSubsection{
\metroset{color/background=dark}
\frame[plain,c]{
  \begin{center}
  \begin{minipage}{25em}
    \usebeamercolor[fg]{section title}
    \usebeamerfont{section title}
    \insertsubsection\\[-1ex]
    \usebeamertemplate*{progress bar in section page}
  \end{minipage}
  \end{center}
}
\metroset{color/background=light}
}
%%%%% end mtheme

\setbeamertemplate{frametitle continuation}[from second]
\setbeamertemplate{bibliography item}[book]

\usepackage{xeCJK}
%\setCJKsansfont[BoldFont=Kozuka Gothic Pro]{Kozuka Gothic Pro L}
\setCJKsansfont{IPAGothic}
% \newCJKfontfamily{\xiheifont}[BoldFont=STHeiti]{STXihei}
\newCJKfontfamily{\xiheifont}{WenQuanYi Micro Hei}

\usetikzlibrary{arrows}
\usetikzlibrary{chains}
\usepackage{tikz-qtree}
\usepackage{multicol}


\usepackage{expex}
%\lingset{glhangindent=2em,glspace=1em,aboveexskip=0pt,belowexskip=0pt,aboveglftskip=-3pt,extraglskip=3pt} %v0.1
%\lingset{exskip=0pt,interpartskip=-3pt,belowpreambleskip=-3pt,belowglpreambleskip=-3pt,aboveglftskip=-3pt,extraglskip=3pt,glhangstyle=none}
\usepackage{relsize}
\usepackage{booktabs,tabularx}
%\usepackage{textcomp}
\usepackage{listings}
\lstset{basicstyle=\ttfamily,breaklines=true,breakatwhitespace=true,
keywordstyle={\color{NavyBlue}\bfseries}, showstringspaces=false,
commentstyle={\color{PaleVioletRed4}},
emphstyle={\color{OliveGreen}\bfseries}
}

\usepackage{algorithmic}
\renewcommand{\algorithmiccomment}[1]{\alert{/* #1 */}}

\usetikzlibrary{shapes.multipart}
\usetikzlibrary{positioning}
\usetikzlibrary{arrows.meta}

\makeatletter
\pgfarrowsdeclare{crow's foot}{crow's foot}
{
  \pgfarrowsleftextend{+-.5\pgflinewidth}%
  \pgfarrowsrightextend{+.5\pgflinewidth}%
}
{
  \pgfutil@tempdima=0.5pt%
  \advance\pgfutil@tempdima by.25\pgflinewidth%
  \pgfsetdash{}{+0pt}%
  \pgfsetmiterjoin%
  \pgfpathmoveto{\pgfqpoint{0pt}{-6\pgfutil@tempdima}}%
  \pgfpathlineto{\pgfqpoint{-6\pgfutil@tempdima}{0pt}}%
  \pgfpathlineto{\pgfqpoint{0pt}{6\pgfutil@tempdima}}%
  \pgfusepathqstroke%
}

\usepackage[backend=biber,style=apa]{biblatex}
\DeclareLanguageMapping{british}{british-apa}
\renewcommand{\finalnamedelim}{and}
\renewcommand{\bibfont}{\small}
\setlength{\bibhang}{1em}
\setlength{\bibitemsep}{1ex}
\bibliography{refs}
\renewcommand{\UrlFont}{\ttfamily}

\usepackage[os=win]{menukeys}


\title{Diretrizes para Avaliação de Ecossistemas}

\subtitle{Fatores sociais e suas influências}

\date{12 de Abril de 2017}
\author{Vinicius Schettino}
\institute{
Departamento de Ciência da Computação\\
Instituto de Ciências Exatas (ICE)\\
Universidade Federal de Juiz de Fora
}

\begin{document}

\maketitle

\begin{frame}[label=LO]
\frametitle{Contexto}

\begin{itemize}
\item Evolução de Linhas de Produto de Sofware para ECOS
\item Revisão e Potencialização dos desafios da ES
\item Poucos problemas novos; Porém, nova dimensão dos conhecidos
\item ECOS são vitais para milhões de pessoas e organizações
\end{itemize}
\end{frame}

\begin{frame}{Motivação}
  Os principais objetivos ao se avaliar um ECOS são:

	\begin{itemize}
		\item Observar a atratividade sob determinados aspectos
		\item Escolher entre concorrentes
		\item Inferir o potêncial e os riscos
	\end{itemize}
\end{frame}


\begin{frame}{Definição}
  Os aspectos sociais se referem a interação entre os agentes participantes:
	\begin{itemize}
		\item Descrevem como os participantes se comunicam
		\item Modelam a manutenção e o monitoramento destas relações
		\item Descrevem as informações relevantes ao contexto que podem ser observadas
	\end{itemize}
\end{frame}

\begin{frame}{Definição}
  Os aspectos sociais são análogos aos ecossistemas biológicos e sociais; Pode-se perguntar:
	\begin{itemize}
		\item Porque os envolvidos interagem desta maneira?
		\item Qual a capacidade e o engajamento dos envolvidos para promoção do ECOS?
		\item Qual conhecimento é obtido através das interações?
	\end{itemize}
\end{frame}

\begin{frame}{Workflow de avaliação}
  A avaliação destes aspectos deve ser feita de seguinte maneira:
	\begin{enumerate}
		\item Modelar os atores e o conhecimento do ECOS
		\item Estabelecer qual é o ambiente de apoio à rede
		\item Calcular fatores sociais do ECOS
	\end{enumerate}
\end{frame}

\begin{frame}{1. Modelar os atores e o conhecimento}
	\begin{itemize}
		\item É preciso identificar quem são os atores e quais são os principais artefatos
    \begin{enumerate}
        \item \textbf{Atores}: Stakeholders, desenvolvedores, jornalistas, etc
        \item \textbf{Artefatos}: Fedback, código, diagramas, requisitos, etc
    \end{enumerate}
		\item Como eles interagem
    \begin{enumerate}
        \item \textbf{Possuído por}: artefato -> proprietário
        \item \textbf{Interessado} em: Atores interessados na evolução de determinado artefato
        \item \textbf{Depende de}: Relação de dependência entre dois artefatos
        \item \textbf{membro de}: Agrupa um conjunto de atores em razão de um interesse comum (cluster)
    \end{enumerate}
	\end{itemize}
\end{frame}


\begin{frame}{2. Estabelecer o Ambiente de Apoio}
	\begin{itemize}
		\item Quais são os elementos de interação?
      \begin{enumerate}
          \item \textbf{Perfil}: Registro público do elemento
          \item \textbf{Mural}: Comentários anexados em atores ou artefatos
          \item \textbf{Compartilhamento de Dados}: download e upload de informações
          \item \textbf{Novas Comunicações}: recebimento de atualizações sobre elementos de interesse
          \item \textbf{Agrupamento}: Clusterização de atores e conhecimentos
          \item \textbf{Pesquisa}: busca por atores e conhecimentos
          \item \textbf{Sugestões}: Inferência de relacionamentos de interesse entre atores e outros atores ou conhecimentos
          \item \textbf{Mensagens}: Comunicação pessoal entre os atores
      \end{enumerate}
	\end{itemize}
\end{frame}

\begin{frame}{3. Calcular fatores sociais}
  \begin{itemize}
    \item Apoiar decisões estratégicas
    \item Registro da dimensão temporal
    \item Influencia:
      \begin{enumerate}
        \item Mudanças na interface
        \item extensão de artefatos
        \item mudanças de responsabilidade entre atores
        \item descoberta automática de conhecimento (e suas relações)
      \end{enumerate}
      \item Observa-se os fatores relevantes:
      \begin{enumerate}
        \item Interação
        \item Utilidade
        \item Reputação
        \item Promoção
        \item Contribuição
        \item Recomendação
      \end{enumerate}
  \end{itemize}
\end{frame}


\begin{frame}[allowframebreaks]
\frametitle{Bibliography}


\nocite{*}
\printbibliography[heading=none]
WERNER, C.M.L., SANTOS, R.P., 2012, “Ecossistemas de Software: Estágio Atual,
Direções de Pesquisa e a Prática na Indústria de Software” In: Proceedings of
the XV Ibero-American Conference on Software Engineering, Buenos Aires,
Argentina.

dos Santos, Rodrigo Pereira. ENGENHARIA E GERENCIAMENTO DE ECOSSISTEMAS DE SOFTWARE. Diss. Universidade Federal do Rio de Janeiro, 2013.

Manikas, Konstantinos. "Revisiting software ecosystems research: A longitudinal literature study." Journal of Systems and Software 117 (2016): 84-103.

\end{frame}

\end{document}
